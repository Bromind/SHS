\documentclass[twoside]{report}
\usepackage[utf8]{inputenc}
\usepackage{VassorTitle}
\usepackage[super]{nth}
\usepackage{todonotes}

\institution{École polytechnique fédérale de Lausanne}
\project{\nth{1} semester report}
\supervisor{Pr.~\textsc{Vinck}}



% Document description
\title{Proposal for a methodology of startup analysis}
\author{Giulia \textsc{Ferrari}\\Martin \textsc{Vassor}}

% Beginning document
\begin{document}
\maketitle
\begin{abstract}
	This report is made for the HUM-428(a) course entitled \emph{Sciences,~technology~and~society}, told by Pr.~\textsc{Vinck} with Richard~\textsc{Marion} and Alexandre~\textsc{Camus} as lecturer assistants at the \textsc{École Polytechnique Fédérale de Lausanne}. Its goal is to conclude the fall semester lecture and provide a link to the spring semester. For this purpose, it first sums up the knowledge we acquired during 
\end{abstract}
\tableofcontents
\listoffigures
\listoftables
\chapter{Learned knowledge}
\section{Introduction}
\chapter{Description of the startup}
\section{SqeedTime}
SqeedTime is a new startup from Lausanne. Its purpose is to provide a smartphone application to enforce the link between local shops, restaurants or all kinds of professionals and the potential customers.
\marginpar{
	\begin{itemize}
		\item Spontaneous meetings
	\end{itemize}
}



\paragraph{Organization}
At first, SqeedTime was founded by Paul-Edgar \textsc{Levy} and Timothée \textsc{Barghouth}, two law students. Now, SqeeTime consists of a team of five members.
\marginpar{
	Demander les autres
}
\paragraph{History and original idea}
During summer 2014, both entrepreneurs observed that even with social networks, they couldn't organize a spontaneous meeting with friend. After this statement, they decided to create such a tool, which finally leads to the application to be released soon.
\section{Product}
\section{Activity domain}
\chapter{Kinds of uncertainty}
We are trying to provide a first raw classification of uncertainties we are expected to be confronted to. The role of this cahpter is not to make an exhaustive list, since it would constraint our perception of the startup, but mainly to have different sources of incertitudes, to be able to analyse if we encounter them.
\section{Audience}
\section{Finances}
\section{Team work}

\chapter{Working methodology}
\begin{itemize}
	\item Study the activity domain.
	\item Find an other company in a more advanced stage.
\end{itemize}
\chapter{Conclusion}
\appendix
\end{document}
