\documentclass[twoside]{report}
\usepackage[utf8]{inputenc}
\usepackage{VassorTitle}
\usepackage[super]{nth}
\usepackage{lettrine}
\usepackage{xstring}
\usepackage{todonotes}

\institution{École polytechnique fédérale de Lausanne}
\project{\nth{1} semester report}
\supervisor{Pr.~\textsc{Vinck}}

%\renewcommand\paragraph[1]
%{
%	\lettrine{\StrLeft #1 {1}}{\StrGobbleLeft #1 {1}}
%}
% TODO : Lettrine paragraphe



% Document description
\title{Proposal for a methodology of startup analysis}
\author{Giulia \textsc{Ferrari}\\Martin \textsc{Vassor}}

% Beginning document
\begin{document}
\maketitle
\begin{abstract}
	This report is made for the HUM-428(a) course entitled \emph{Sciences,~technology~and~society}, told by Pr.~\textsc{Vinck} with Richard~\textsc{Marion} and Alexandre~\textsc{Camus} as lecturer assistants at the \textsc{École Polytechnique Fédérale de Lausanne}. Its goal is to conclude the fall semester lecture and provide a link to the spring semester. For this purpose, it first sums up the knowledge we acquired during 
\end{abstract}
\tableofcontents
\listoffigures
\listoftables
\chapter{Learned knowledge}
\section{Introduction}
\chapter{Description of the startup}
\section{SqeedTime}
SqeedTime is a new startup from Lausanne. Its purpose is to provide a smartphone application to enforce the link between local shops, restaurants or all kinds of professionals and the potential customers.
\marginpar{
	\begin{itemize}
		\item Spontaneous meetings
	\end{itemize}
}



\paragraph{Organization}
At first, SqeedTime was founded by Paul-Edgar \textsc{Levy} and Timothée \textsc{Barghouth}, two law students. Now, SqeeTime consists of a team of five members.
\marginpar{
	Demander les autres
}
\paragraph{History and original idea}
During summer 2014, both entrepreneurs observed that even with social networks, they couldn't organize a spontaneous meeting with friend. After this statement, they decided to create such a tool, which finally leads to the application to be released soon.
\section{Product}
\section{Activity domain}
\chapter{Kinds of uncertainty}
We are trying to provide a first raw classification of uncertainties we are expected to be confronted to. The role of this cahpter is not to make an exhaustive list, since it would constraint our perception of the startup, but mainly to have different sources of incertitudes, to be able to analyse if we encounter them.
\section{Audience}
\section{Finances}
\section{Team work}

\chapter{Working methodology}
\begin{itemize}
	\item Study the activity domain.
	\item Find an other company in a more advanced stage.
\end{itemize}
\section{Objectives and general approaches}
\paragraph{Introduction }During the second semester, our goal is to observe the startup SqeedTime, to highlight some correlations between some signals, some events or tendencies. We will have to approaches of this problem.

\paragraph{Precise observation}Our first approach is to make some hypothesis on which signals could be correlated or not, and to gather precise datas to verify these hypothesis or not. This approach requires two steps : 
\begin{itemize}
	\item We have to be able to formulate precise and simple enough hypothesis. Making too general or too complex hypothesis would lead not to search for specific behavior, leading in the failure of the verification. This approach does not aim for an exhaustive model of correlations\footnote{If specialists of the field still encounter difficulties, it is definitivelly not possible for us, in such a short amount of time and of knowledge.}.
	\item Once these points are well defined, we then have to develop specific tools to evaluate the signals. Here, the goal is to develop an high-level framework for data analysis, i.e. to guess which metrics would be significant for the evaluation, then to work on a way to compute this metric, and finally to have a relevant way to mesure the data needed for these calculations.
\end{itemize}
Notice that this approach is really uncertain, which is kind of ironic. Even if it is possible to make a good hypothesis, which then appears to be relevant with time, it is possible that this approach leads to a complete fail as well. Also, even if the developped framework appears to be relevant in our case, one should test it with other startups to verify its relevance.

This approach requires an high amount of work before starting the actual observation, but then it only consists in gathering data and verifing or invalidate the hypothesis.
\paragraph{General observation}Our second approach is the exact opposite of the first one. This time, we observe the startup as an external observer, that is we consider the startup globally. After observing the whole startup for a given amount of time, we try to get important events, and finally to identify the causes of these events. This approach does not focus on events \emph{a priori}. With this approach, we will have global tendencies, global causes for events, but we won't focus on a single event, or on single causes. It is more a reverse engineering of the uncertainty problem. 
\begin{itemize}
	\item First, we have to define which data we want to collect, which rate, etc. The more types of data and the more fined-grained, the more precise will be the analysis, but the harder to collect, store, and analyse. We have to make a good trade-off, with these different parameters.
	\item Once our data set is collected, we have a huge step of analysis. For this approach, we'll have to develop the relevant tools \emph{a posteriori}. 
\end{itemize}
This approach needs good heuristics on the parameters of data collection, since it is clearly not possible to get all the information. Making heuristics may lead to holes in the data fabric, network we are building. These hole may be corrected be gathering wide enough events, at a cost of not being able to detect small relations.

This approach requires an high amount of work after the data collection, to understand the causallity processus which happened during the observation.
\section{Tools for the first method}
\subsection{Hypothesis}
\subsection{Metrics}
\subsection{Analysis}
\subsection{Data requirements}

\section{Heuristics for the second method}
\subsection{Data set}
\subsection{Data rate}
\chapter{Conclusion}
\appendix
\end{document}
