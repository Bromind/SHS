\documentclass{article}
\usepackage[utf8]{inputenc}

\title{Esquisse Rapport Final}

\author{Martin Vassor}


\begin{document}
\maketitle

\section{Start-up}

Lors du forum EPFL, j'ai rencontré Paul-Edgar Levy de la start-up Sqeedtime. Sqeedtime developpe une application permettant à ses utilisateurs d'organiser des évènements. L'idée originale de cette application est de rapprocher les particuliers des entreprises locales. Typiquement, une entreprise peut proposer des avantages (tels que des reductions, etc.) aux utilisateurs de l'application. Ainsi, lorsqu'un utilisateur veut, par exemple, trouver un bowling, il verra les bowlings partenaires. 

Le choix de cette start-up s'est fait en fonction des points suivants : 
\begin{itemize}
	\item La start-up travaille dans le domaine de l'informatique, qui est mon domaine d'étude.
	\item La sortie de l'application est prévue en novembre, je pourrai donc suivre cette période.
	\item La start-up est implantée sur le campus de l'EPFL, ce qui permet de pouvoir suivre facilement les évènements.
\end{itemize}


\section{Démarche}
Dans un premier temps, j'ai rendez-vous avec Paul-Edgar Levy lundi 2 novembre, je vais d'abord essayer d'identifier les différents points qui ont été soulevés en cours et lors des rencontres avec les intervenants. En parallèle, je vais essayer de suivre de manière systématique les évènements suivant la sortie au grand public de l'application.

Dans un second temps, après avoir pu identifier les points d'incertitude, l'idée est d'identifier si ces points ont été rencontrés par les intervenants du 1er semestre. Si c'est le cas, regarder comment ceux-ci s'en sont sortis. S'il s'agit de nouveaux points, j'essayerai de trouver de la documentations sur comment les gérer.

\end{document}
